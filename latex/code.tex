\section{Software Architecture and Implementation}
Detailed in this section is the method and reason to our software architecture. In particular, we aimed for our architecture to be stateless with respect to the current operation, to maximize efficiency of joystick checks (which are normally slow in RobotC), and to allow dynamic and easy reassignment of both controller information during teleop, as well as both movement sequencing and heuristic decision trees during autonomous.

We have chosen a particular structure for our code. We have rewritten \lstinline{JoystickDriver.c}{} to better express our needs. By removing superfluous tasks from the driver, we maintained functionality while increasing our efficiency by approximately threefold. We have created a set of utility headers: \lstinline{teleoputils.h}{}, \lstinline{autoutils.h}{}, and \lstinline{sharedutils.h}{} to address the need for macros, \lstinline{#define}{} statements, and standardized functions. Both \lstinline{teleoputils.h}{} and \lstinline{autoutils.h}{} import \lstinline{sharedutils.h}{}, so that file is never explicitly included in the main code body. Drivers were pulled from HiTechnic's online library for 3rd party sensors.

\subsection{Teleoperated Mode}

Teleoperated mode has the following requirements: \begin{enumerate}
	\item{Smooth arcade driving}
	\item{Easy reassignment of buttons}
	\item{Stateless control of motors and robot state}
	\item{Efficient button control and loop checking}
	\item{\textbf{Must} use only one controller}
\end{enumerate}

In order to implement this effectively, we have implemented several macros. These macros allow us to later set the powers of the motors without much effort. 

\subsubsection{Drive Code and Reasoning}

Our main body of code is run through the following:

\begin{lstlisting}[tabsize=4]
task main() {
	while(true) {
		getJoystickSettings(joystick);
		checkJoystickButtons();
		setLeftMotors(powscl(JOY_Y1)-powscl(JOY_X1)/1.75);
		setRightMotors(powscl(JOY_Y1)+powscl(JOY_X1)/1.75);
	}
}
\end{lstlisting}

First, grab the current joystick configuration from the controllers. Then, check to see if any buttons have changed (\lstinline{checkJoystickButtons();}{}). Then, set the motor powers for arcade drive. 

\paragraph{Power scaling} The \lstinline{powscl(int)}{} function's definition is intended to compensate for the large deadband range which occurs under standard drive conditions. The controller's user really only needs two ranges: a high-precision, low power range near zero, and a low-precision, high power range near the maxima. While an exponential function could be used, it is much slower, and much more hard to tune. Instead, we draw two lines: a shallow slope for the first segment, then a large slope for the second segment of the controller. This provides both high precision and high power where needed. As the driver does not generally use the range in $[60,85]\%$, there is no concern about the nonlinearity. The function is defined as follows:

\begin{lstlisting}[tabsize=4]
float powscl(int xz) {
	float sign = (float)sgn(xz);
	float x = abs(xz)/128.0;
	if(x < DISTA)
		return 100* sign * (x*SLOPE);
	else
		return 100* sign * ((DISTA*SLOPE*(x-1.0) - x + DISTA) / (DISTA - 1.0));
}
\end{lstlisting}

\paragraph{Compensation for the Old and New Joystick Configuration} It is necessary to compensate for both the old and new controller configurations. As the controller has been updated, the buttons have changed - however, competition rules premit the use of both controllers. Therefore, we must be able to accomodate this change if necessary. We have done so through the use of a define statement: if we have \lstinline{#define ALTLOG}{}, then we switch to the old button layout.

\paragraph{Button Press Checking Requires a Stateless Organization} In order to easily and effectively change the functionality of the controller, a particular design was implemented:

\begin{lstlisting}[tabsize=4]
void invokeButton(int button, bool pressed) {
	switch(button) {
		case JOY_X:  if(pressed) {servo[servoL1] = 156; servo[servoL2] = 26;} else {} break;
		case JOY_Y:  if(pressed) {servo[servoL1] = 120; servo[servoL2] = 40;} else {} break;
		case JOY_A:  if(pressed) {} else {} break;
		case JOY_B:  if(pressed) {motor[mSpin] = 100;} else {motor[mSpin] = 0;} break;
		case JOY_RB: if(pressed) {setArmMotors(100);}  else {setArmMotors(0);} break;
		case JOY_LB: if(pressed) {setArmMotors(-100);} else {setArmMotors(0);} break;
		case JOY_R3: if(pressed) {} else {} break;
		case JOY_L3: if(pressed) {} else {} break;
	}
}

bool t[8];
void checkJoystickButtons() {
	for(int i = 0; i < 8; i++) {
		if(joy1Btn(i) != t[i]) {
			invokeButton(i, !t[i]);
			t[i] = !t[i]; 
		}
	}
}
\end{lstlisting}

This may appear confusing at first, however, there are a couple points: \lstinline{checkJoystickButtons()}{} is actually called from the main loop. It simply checks to see what buttons have changed on the controller, and calls the appropriate \lstinline{invokeButton(int, bool)}{} arguments. In doing so, we can determine exact behavior on button presses with ease. As our robot is very simple, we do not need more than a handful of buttons, so most of them remain unassigned.

\newpage \paragraph{Code Optimizations} Although this method works for determining which button is being pressed, we quickly found it is not the most elegant way of doing so. First off, we use a \lstinline{bool[]}{} to hold the state of each button. This seems intuitive, but the RobotC compiler actually creates an entire \lstinline{char}{} to hold either \lstinline{true}{} or \lstinline{false}{} for every \lstinline{bool}{}; this is a waste of memory. The other issue can be found in the processing structure. Each iteration of the program checks all possible button conditions; this is a waste of processing time if there has been no change since the last iteration.

\begin{lstlisting}[tabsize=4]
short btn = JOY_BTN;//local store = live store, initially
void checkJoystickButtons() {
	if(btn == JOY_BTN) return;//checksum
	for(short i = 11; i >= 0; i--) {
		if((btn>>i) ^ (JOY_BTN>>i)) {//check each button for a change
			invokeButton(i, ((btn & (1 << i)) == 0));//trigger event (#, down|up)
			btn ^= 1<<i;//mirror changes in local store
		}
	}
}
\end{lstlisting}

We solve the former issue by means of storing each button state in a single bit of data, reducing our memory footprint. Teams are encouraged to call the \lstinline{joy1Btn(int)}{} function each time they wish to check the state of a particular button, but this can become cumbersome of one wishes check multiple buttons in real-time. The ``JoystickDriver.c'' file that we must use for field communications stores each button state in a \lstinline{short}{}; this means that we are capable of running a checksum before we check each button. This not only reduces our time per iteration, but also allows for our robot to be more responsive to joystick changes due to the inherent speed of bitwise operations.

\newpage \subsection{Autonomous Mode}
Autonomous, much like teleop, must meet certain criteria
\newpage \subsection{Algorithms and Cartesian Mathematics}

\subsubsection{Hybrid Localization Using Gyroscopes \& Odometers}
\paragraph{Odometric Data}
We begin by assigning the following constants: 
\[D_{ot}=\frac{\text{Distance}}{\text{odometer tick}}=\pi(\text{wheel diameter})/(\text{ticks/revolution})\]\[ \theta_{ot} = \frac{\theta}{\text{odometer tick}} = \pi\left(\frac{\text{wheel diameter}}{\text{distance between wheels}}\right)/(\text{ticks/revolution})\]
We can calculate $(x_{\mathrm{enc}}, y_{\mathrm{enc}}, \theta_{\mathrm{enc}})$ from the odometer as follows: 
\begin{center}
	\begin{align*}
		\mathrm{d}l &= l^t_{\mathrm{enc}}-l^{t-1}_{\mathrm{enc}} \\
		\mathrm{d}r &= r^t_{\mathrm{enc}}-r^{t-1}_{\mathrm{enc}} \\
		\mathrm{d}D &= \frac{1}{2}(\mathrm{d}l+\mathrm{d}r)D_{ot} \\
		\mathrm{d}x_{\mathrm{enc}} &= \mathrm{d}D\cos(\theta^t_{enc}) \\
		\mathrm{d}y_{\mathrm{enc}} &= \mathrm{d}D\sin(\theta^t_{enc}) \\
		\mathrm{d}\theta_{\mathrm{enc}} &= (\mathrm{d}r-\mathrm{d}l)\theta_{ot} \\
		x_{\mathrm{enc}} &= x^{t-1}_{\mathrm{enc}} + \mathrm{d}x_{\mathrm{enc}} \\
		y_{\mathrm{enc}} &= y^{t-1}_{\mathrm{enc}} + \mathrm{d}y{\mathrm{enc}} \\
		\theta_{\mathrm{enc}} &= \theta^{t-1}_{\mathrm{enc}} + \mathrm{d}\theta_{\mathrm{enc}}
	\end{align*}
\end{center}

$l$ denotes the left side of the robot, and $r$ denotes the right side of the robot. $D$ denotes the distance. 

\paragraph{Localization Algorithm}
The robots motor controller calculates position and orientation $(x_{\mathrm{enc}}, y_{\mathrm{enc}}, \theta_{\mathrm{enc}})$ from encoder ticks and sends the data to an on-board computer. The mounted gyroscope communicates with a gyro driver which integrates the rate values into an absolute angle ($\theta_{\mathrm{gyro}}$). Global position $(x_{\mathrm{rbt}}, y_{\mathrm{rbt}})$ is found by transforming the translation vector from encoder space to gyroscope space. Global angle $(\theta_{\mathrm{rbt}})$ is the gyro angle ($\theta_{\mathrm{gyro}}$). The following describes the computation as an iterative algorithm: 

\begin{center}
	\begin{align*}
		\mathrm{d}x&=x^{t}_{\mathrm{enc}}-x^{t-1}_{\mathrm{enc}} \\
		\mathrm{d}y&=y^{t}_{\mathrm{enc}}-y^{t-1}_{\mathrm{enc}} \\
		\mathrm{d}\theta&=\theta^t_{\mathrm{gyro}}-\theta^{t}_{\mathrm{enc}} \\
		x^t_{\mathrm{rbt}}&=x^{t-1}_{\mathrm{rbt}}+\cos(\mathrm{d}\theta)\mathrm{d}x-\sin(\mathrm{d}\theta)\mathrm{d}y \\
		y^t_{\mathrm{rbt}}&=y^{t-1}_{\mathrm{rbt}}+\sin(\mathrm{d}\theta)\mathrm{d}x+\cos(\mathrm{d}\theta)\mathrm{d}y \\
		\theta^t_{\mathrm{rbt}}&=\theta^t_{\mathrm{gyro}}
	\end{align*}
\end{center}
