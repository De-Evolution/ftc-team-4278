\documentclass{article}
\usepackage[margin=1in]{geometry}
\usepackage{amsmath}
\usepackage[T1]{fontenc}
\title{\textbf{Engineering Journal}}
\author{FTC Team 4278\\de.evolution}

\setlength{\parskip}{4pt}

\begin{document}\maketitle\tableofcontents\newpage

\section{Preseason Overview -- Meeting 1: 2013-08-31}
We held a preseason meeting in order to go over scheduling, recruitment of new members, and hold a quick review of what we learned from our last season. One of the main items we discussed was our scheduling, and as a team we decided to take an aggressive approach towards our first couple of regional competitions in order to secure a place in St. Louis. Because of this, we may need to cut back on scoring the maximum number of points and instead focus on scoring a high, yet consistent number of points. We found out that we need to secure our electronics and work with the field control issues that we were issues. We also plan on placing a much larger emphasis on the CAD design of our robot than we have in our two previous years as it is an efficient way to quickly discover and troubleshoot problems prior to building the system. 

Our final goal is to have two weeks of drive practice before our first regional on December 14th, important for both driver and coaches, to figure out the timing of the game, as well as things that we can or cannot do in game. Organization of the team this year will be facilitated through the use of Google Groups, which will allow all the team members and their parents to be easily contacted for meetings, and will hopefully foster some discussion over build design or game strategy. 

We also decided to meet a few hours after the game was released next week so team members could think about strategies ahead of time and add to the strategy discussion of our first meeting of the season. The meeting was fairly short, but got the team in the right mindset for the upcoming season, and got everyone excited for the  new season!

\newpage
\section{Initial Design -- Meeting 2: 2013-09-07}
The team decided last week to meet a few hours after the game video and rules were released, so by the time our meeting started, everyone had an idea of how they thought the robot should work, and be built. Fortunately, most team members were on the same page and wanted to focus on scoring as many blocks into the baskets as possible, in a manner that would allow for a integration in the lifting and scoring mechanism. We decided to focus on every aspect of the game for our qualifiers, since it seems like we could consolidate all of the mechanisms into very few.

After a fairly long strategy discussion, some ideas were thrown around as to how to pick up and score the blocks, since we always have a difficult time getting our game pieces, and a rough idea of a mechanism was developed that would use a roller system and a block hopper that would pick up and score the blocks. There is still significant discussion considering the way we will go about constructing a lifting mechanism, with different arguments for and against a rotating arm and a more standard but perhaps less efficient forklift, similar to ``Ring It Up!''. Some of our team members have some experience with constructing forklifts from prior FTC seasons, so we have some idea of what kind of issues we might come across with that kind of design. One thing we discussed was making sure we either construct our purchase our materials with a little more regard for quality than we have in previous years, but we wanted to focus on simplicity this season as it worked out very well for us three years ago, and the ideas we are thinking of currently all seem to fit this idea.

Members of the team have assumed different jobs to complete before the next meeting, such as researching lifting mechanisms, getting a BOM for the field and purchasing the materials, and researching the specifications of an IR beacon and sensor, since the team also decided that scoring the block during autonomous is absolutely imperative to the outcome of this game. It is essentially free points that even a defensive robot cannot stop. As of right now we are choosing to focus the endgame period, since lifting and raising the flag are relatively simple tasks. 

We are considering purchasing the AndyMark field, as it would provide us a standardized field and a better replication of the interaction of the robot with the field during competition. 
\end{document}